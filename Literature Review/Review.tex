\documentclass{article}
\usepackage[utf8]{inputenc}

\title{Literature-Review BE-project}
\author{pranavparwate10 }
\date{November 2020}

\begin{document}

\maketitle

\section{Introduction}

Hubble Legacy Archive or HLA is a project which endeavours to complement the Hubble Space Telescope by augmenting the HST Data Archive and providing superior browsing and searching capabilities. A large amount of raw images remain unprocessed in the HLA, never seen by a human eye. These raw images are typically low resolution, black and white and unfit to work with in today's day and age. It takes hundreds of hours to process them.


\section{Hubble Legacy Archive}


In the past decade, astronomical research is extensively using large catalogs which are searchable which is made possible due to advances in computer technology and databases. The biggest challenge that is faced today is to convert this  large unstructured database into a comprehensive catalog. Advances in relational databases technology has made it efficient to create and store and search large catalogs. 

Sloan Digital Sky Survey or SDSS, was one such catalog. It uniformly observed the regions of the sky in a certain filter band at regular intervals of time. The HST however, has targeted only particular sources.

The Hubble Space Telescope (often referred to as HST or Hubble) is a space telescope that was launched into low Earth orbit in 1990 and remains in operation. It was not the first space telescope, but it is one of the largest and most versatile, well known both as a vital research tool and as a public relations boon for astronomy.

The Hubble Space Telescope archive was an archive for the Hubble Space Telescope which was launched in low Earth orbit back in 1990. It is an important tool for research and is the one of the biggest and most versatile telescopes which is active today.
The Hubble Legacy Archive (HLA) endeavours to create calibrated science data from the Hubble Space Telescope archive and make them accessible via user-friendly and Virtual Observatory (VO) compatible interfaces. 



\end{document}
